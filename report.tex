\documentclass[12pt]{article}

\usepackage{amsmath}

\usepackage{graphicx}

\usepackage{hyperref}

\usepackage[utf8]{inputenc}


\title{Credit Card Fraud Detection}

\author{Wen Sun}

\date{03/02/2024}

\begin{document}

\maketitle

\section{Introduction}
The project is to recognize fraudulent credit card transactions so that customers are not charged for items that they did not 
purchase. This project contains two parts: 

    \begin{itemize}
    \item Predict if a transaction is fraudulent or not
    \item Use data streaming techniques to see how many fraudulent transactions happen in the last k seconds
    \end{itemize}

\section{Dataset}
    This dataset contains transactions made by credit cards in September 2013 by European cardholders.\\
    This dataset presents transactions that occurred in two days, where we have 492 frauds out of 284,807 transactions.\\ 
    This dataset is very unbalanced, since the positive class (frauds) count for 0.17\% of all transactions.\\
    It contains only numerical input (V1,V2,...,V28) based on a PCA transformation. Due to confidentially issues, we do not know
    the meaning of these features. "Time" and "Amount" variables are not transformed by the PCA. "Time" contains the seconds elapsed
    between transaction and the first transaction in the dataset. And the 'Amount' is the transaction amount. The target variable is "Class", and
    it takes value 1 as fraud and 0 otherwise. 
\section{Data processing}
    Because the 'Amount' features are not transformed while others do. I scaled the "Amount" variable by applying 
\section{Data Prediction}
\subsection{Original results}

\subsection{Imbalanced dataset}
As I mentioned above, this dataset is very imbalanced, as the positive class (frauds) only count for 0.17\% of all transactions. In other words, the ratio between the fraud and non-fraud is 1: 500.\\
So, our prediction will have a bias to predict the class as non-fraud because it is the majority class. 

\end{document}